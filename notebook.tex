\documentclass[10pt,a4paper]{article}

\usepackage[landscape,margin=0.1in]{geometry}   
\usepackage{multicol}                 
\usepackage{hyperref}                 
\usepackage{parskip}                  
\usepackage{minted}
\usepackage{graphicx}

\setlength{\columnsep}{12pt}


\setminted{
  fontsize=\footnotesize,
  breaklines,
  breakanywhere,
  tabsize=2
}

\begin{document}



\begin{titlepage}
  \centering

  \vspace*{2cm}
  {\Huge\bfseries ICPC Team Notebook\par}
  \vspace{1cm}

  {\Large {typedef unsigned long long uwu}\par}
  {\Large {Sorbonne Université}\par}
  {\Large SWERC 2025\par}
  \vspace{1.5cm}

  \rule{\linewidth}{1.5pt}\par
  \vspace{0.5cm}
  \rule{\linewidth}{0.5pt}\par
  \vspace{1.5cm}

  {\LARGE \itshape A curated reference of algorithms and data structures\par}
  \vspace{1cm}

  \includegraphics[width=0.2 \textwidth]{img/logo.png}\par
  \vspace{1cm}

  {\large \today\par}

  \clearpage
\end{titlepage}


\clearpage
\tableofcontents

\twocolumn

\section{Data Structures}

\subsection{Fenwick Tree}

Point update, prefix and range sum, $O(log n)$, $O(n)$ build
\begin{minted}{cpp}
struct Fenwick {
    int n; vector<ll> t; //using long long, int might overflow.
    Fenwick(vector<ll>& a): n(a.size()), t(n+1,0) {
        for(int i = 1; i <=n; i++) { // Builds tree from array in O(n)
            t[i] += a[i-1];
            int p = i+(i&-i);
            if(p<=n) t[p] += t[i];
        }
    }
    void add(int i,long long v){
        for(;i<=n;i+=i&-i) t[i]+=v;
    }
    long long sum(int i){
        long long r=0;
        for(; i>0 ;i -= i&-i) r += t[i];
        return r;
    }
    long long sum(int l,int r){ return sum(r)-sum(l-1); }
};
\end{minted}

\subsection{union find}

\textbf{Complexity:} nearly O(1)
\begin{minted}{cpp}
struct union_find{
    vector<int> rank, parent;
    union_find(int n){
        rank.resize(n, 0); parent.resize(n);
        for (int i = 0; i < n; i++) parent[i] = i;
    }
    int find(int i){
        int root = parent[i];
        if (parent[root] != root) return parent[i] = find(root);
        return root;
    }
    void unite(int x, int y) {
        int xRoot = find(x);
        int yRoot = find(y);
        if (xRoot == yRoot) return;
        if (rank[xRoot] < rank[yRoot]) parent[xRoot] = yRoot;
        else if (rank[yRoot] < rank[xRoot]) parent[yRoot] = xRoot;
        else{
            parent[yRoot] = xRoot;
            rank[xRoot]++;
        }
    }
} union_find;
\end{minted}

\section{Graph Algorithms}

\subsection{Max Flow Min Cut}

\textbf{Dinic's max flow} $O(EV^2)$ worst case, often $O(E\sqrt(V))$ in practice.
Edges from reachable nodes after flow form a min cut.
\begin{minted}{cpp}
struct FlowEdge {
    int v, u;
    long long cap, flow = 0;
    FlowEdge(int v, int u, long long cap) : v(v), u(u), cap(cap) {}
};
struct Dinic {
    const long long flow_inf = 1e18;
    vector<FlowEdge> edges;
    vector<vi> adj;
    int n, m = 0;
    int s, t;
    vi level, ptr;
    queue<int> q;
    Dinic(int n, int s, int t) : n(n), s(s), t(t) {
        adj.resize(n);level.resize(n);ptr.resize(n);
    }
    void add_edge(int v, int u, long long cap) {
        edges.emplace_back(v, u, cap);
        edges.emplace_back(u, v, 0);
        adj[v].push_back(m);
        adj[u].push_back(m + 1);
        m += 2;
    }
    bool bfs() {
        while (!q.empty()) {
            int v = q.front();
            q.pop();
            for (int id : adj[v]) {
                if (edges[id].cap == edges[id].flow)
                    continue;
                if (level[edges[id].u] != -1)
                    continue;
                level[edges[id].u] = level[v] + 1;
                q.push(edges[id].u);
            }
        }
        return level[t] != -1;
    }
    long long dfs(int v, long long pushed) {
        if (pushed == 0)
            return 0;
        if (v == t)
            return pushed;
        for (int& cid = ptr[v]; cid < (int)adj[v].size(); cid++) {
            int id = adj[v][cid];
            int u = edges[id].u;
            if (level[v] + 1 != level[u])
                continue;
            long long tr = dfs(u, min(pushed, edges[id].cap - edges[id].flow));
            if (tr == 0)
                continue;
            edges[id].flow += tr;
            edges[id ^ 1].flow -= tr;
            return tr;
        }
        return 0;
    }
    long long flow() {
        long long f = 0;
        while (true) {
            fill(level.begin(), level.end(), -1);
            level[s] = 0;
            q.push(s);
            if (!bfs())
                break;
            fill(ptr.begin(), ptr.end(), 0);
            while (long long pushed = dfs(s, flow_inf)) {
                f += pushed;
            }
        }
        return f;
    }
    vector<pii> min_cut_edges() {
        vector<bool> vis(n, false);
        queue<int> q;
        q.push(s);
        vis[s] = true;
        while (!q.empty()) {
            int v = q.front(); q.pop();
            for (int id : adj[v]) {
                auto &e = edges[id];
                if (!vis[e.u] && e.cap > e.flow) {
                    vis[e.u] = true;
                    q.push(e.u);
                }
            }
        }
        vector<pii> cut;
        for (auto &e : edges) {
            if (vis[e.v] && !vis[e.u] && e.cap > 0) {
                cut.push_back({e.v, e.u});
            }
        }
        return cut;
    }
};
\end{minted}

\subsection{Shortest Path}

Find shortest paths from src (no negative weights). O((V+E)logV)
\begin{minted}{cpp}
vi dijkstra(const vector<vector<pii>>& adj, int src) {
    vi dist(adj.size(), INT_MAX);
    priority_queue<pii, vector<pii>, greater<pii>> q;
    dist[src] = 0; q.push({0, src});
    while (!q.empty()) {
        auto [d, u] = q.top(); q.pop();
        if (d != dist[u]) continue;
        for (auto [v, w] : adj[u]) {
            if (d+w < dist[v]) {
                dist[v] = d+w;
                q.push({d+w, v});
            }
        }
    }
    return dist;
}
\end{minted}
Shortest paths from src (handles negative edges). Detects neg cycles. $O(VE)$
\begin{minted}{cpp}
vector<int> bellmanFord(int n, vector<vector<int>>& edges, int src) {
	vector<int> dist(n, INT_MAX);
	dist[src] = 0;
	for (int i = 0; i < n; i++) {
		for (vector<int> edge : edges) {
			int u = edge[0];int v = edge[1];int wt = edge[2];
			if (dist[u] != INT_MAX && dist[u] + wt < dist[v]) {
                if(i == n - 1) return {-1};
                dist[v] = dist[u] + wt;
            }
		}
	}
    return dist;
}
\end{minted}
All-pairs shortest paths (neg edges ok, no neg cycles) $O(V^3)$
$graph[i][i]=0$, $graph[i][j]=w$ if edge $i->j$ else $INT\_MAX$
\begin{minted}{cpp}
vector<vi> floydWarshall(vector<vi> graph) {
    int V = graph.size();
    auto dist = graph;
    for (int k = 0; k < V; ++k)
        for (int i = 0; i < V; ++i)
            for (int j = 0; j < V; ++j)
                if (dist[i][k] < INT_MAX && dist[k][j] < INT_MAX)
                    dist[i][j] = min(dist[i][j], dist[i][k] + dist[k][j]);
    return dist;
}
\end{minted}

\subsection{Toposort}

Topological sort, works for non connected graphs.
\begin{minted}{cpp}
void dfs(int u, vector<vi> &adj, vi &vis, vi &res) {
    vis[u] = 1;
    for (int v : adj[u])
        if (!vis[v])
            dfs(v, adj, vis, res);
    res.push_back(u);
}
vi toposort(vector<vi> &adj) {
    int n = adj.size();
    vi vis(n, 0), res;
    for (int i = 1; i < n; i++)
        if (!vis[i]) dfs(i, adj, vis, res);
    reverse(res.begin(), res.end());
    return res;
}
\end{minted}

\section{Arrays}

\subsection{Inversions}

Count pairs where order flips between arrays. $O(n log n)$
\begin{minted}{cpp}
ll inversions(vi& a, vi& b) {
    int n = a.size();
    unordered_map<int,int> pos;
    for (int i = 0; i < n; i++) pos[b[i]] = i + 1;
    Fenwick t(n); // C.f. Fenwick tree
    ll inv = 0;
    for (int i = 0; i < n; i++) {
        inv += i - t.sum(pos[a[i]]);
        t.add(pos[a[i]], 1);
    }
    return inv;
}
\end{minted}


\end{document}
